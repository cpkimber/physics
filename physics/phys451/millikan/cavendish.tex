%%%%%%%%%%%%%%%%%%%%%%%%%%%%%%%%%%%%%%%%%
% University/School Laboratory Report
% LaTeX Template
% Version 3.1 (25/3/14)
%
%%%%%%%%%%%%%%%%%%%%%%%%%%%%%%%%%%%%%%%%%

%----------------------------------------------------------------------------------------
%	PACKAGES AND DOCUMENT CONFIGURATIONS
%----------------------------------------------------------------------------------------

\documentclass{article}
\usepackage{siunitx} % Provides the \SI{}{} and \si{} command for typesetting SI units
\usepackage{graphicx} % Required for the inclusion of images
\usepackage{natbib} % Required to change bibliography style to APA
\usepackage{amsmath} % Required for some math elements

\setlength\parindent{0pt} % Removes all indentation from paragraphs

\renewcommand{\labelenumi}{\alph{enumi}.} % Make numbering in the enumerate environment by letter rather than number (e.g. section 6)

%\usepackage{times} % Uncomment to use the Times New Roman font

%----------------------------------------------------------------------------------------
%	DOCUMENT INFORMATION
%----------------------------------------------------------------------------------------

\title{Measurement of Universal Gravitational Constant \\ Using a Cavendish Balance \\ PHYS 451} % Title

\author{Cameron \textsc{Kimber}} % Author name

\date{\today} % Date for the report

\begin{document}

\maketitle % Insert the title, author and date

\begin{center}
\begin{tabular}{l r}
Instructor: & Dr. Minschwaner % Instructor/supervisor
\end{tabular}
\end{center}

% If you wish to include an abstract, uncomment the lines below
\begin{abstract}

\end{abstract}
Ever important in the history of physics and in making accurate measurements of masses in the universe
is the universal gravitational constant, $G$. Using a boom arm, suppprting two small masses, supended by a thin, tungsten wire, I
have made a measurement of this value. The system was set to oscillate, changing the capacitance of a
parallel plate capacitor beneath the boom. When the oscillations became small enough, I began to drive them
using a larger boom supporting two larger masses, mounted beneath the system. The force of gravity between the
masses drives the wire-supported boom back and forth. By finding the torsional coefficient of the wire,
the angular displacement of the boom in the forced regime, the distance between the masses, and the masses themselves,
I have determined the value of $G$ to be $5.49 \times 10^{-11} kg m^{3} s^{-2} \pm 7.14 \times 10^{-12} kg m^{3} s^{-2}$.
The accepted value for this constant is $6.67 \times 10^{-11} kg m^{3} s^{-2}$, which lies at $\sigma$ of $1.66$ given
my measurement. I feel, given these results, that the measurement is accurate; there are, however, ways the accuracy of my
value could be improved. I had not properly calibrated the voltmeter reading out the values, so they did not oscillate over 0V.
Additionally, a laser reflecting off the balance arm, projected on a ruler $~1 m$ away was supposed to be centered at $60$ cm on the
ruler, but this too was off by a small fraction. In my data reduction, I did use 0 V and $60$ cm as these values, which may have led
to small discrepancies. Additonally, it would be worthwhile taking more data to ensure the accuracy and drive down the uncertainty in
the results.
\end{document}
