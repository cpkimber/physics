%%%%%%%%%%%%%%%%%%%%%%%%%%%%%%%%%%%%%%%%%
% University/School Laboratory Report
% LaTeX Template
% Version 3.1 (25/3/14)
%
%%%%%%%%%%%%%%%%%%%%%%%%%%%%%%%%%%%%%%%%%

%----------------------------------------------------------------------------------------
%   PACKAGES AND DOCUMENT CONFIGURATIONS
%----------------------------------------------------------------------------------------

\documentclass{article}
\usepackage{graphicx} % Required for the inclusion of images
\usepackage{natbib} % Required to change bibliography style to APA
\usepackage{amsmath} % Required for some math elements

\setlength\parindent{0pt} % Removes all indentation from paragraphs

\renewcommand{\labelenumi}{\alph{enumi}.} % Make numbering in the enumerate environment by letter rather than number (e.g. section 6)

%\usepackage{times} % Uncomment to use the Times New Roman font

%----------------------------------------------------------------------------------------
%   DOCUMENT INFORMATION
%----------------------------------------------------------------------------------------

\title{Finding the Fundamental Charge \\ with the Millikan Oil Drop Experiment\\ PHYS 451} % Title

\author{Cameron \textsc{Kimber}} % Author name

\date{\today} % Date for the report

\begin{document}

\maketitle % Insert the title, author and date

\begin{center}
\begin{tabular}{l r}
Instructor: & Dr. Minschwaner % Instructor/supervisor
\end{tabular}
\end{center}

\newcommand{\qel}{electron charge}
\newcommand{\sel}{uncertainty}

% If you wish to include an abstract, uncomment the lines below
\begin{abstract}
Across most scales of structure in the universe, many interactions of matter are dictated
by electrical forces. Charge, like matter, can only be broken down so far until it is
made solely of fundamental particles. The most common charge we associate with charge
is the electron. All electrons are identical- they carry the same charge, have the same
inertial and gravitational masses, etc. It is important then to characterize electrons,
and measure the value of their charges- which in turn suggests we have the proton charge
as well. To do this, I sprayed microcopic bits of oil, which had become electrically
charged by jostling over one another, in between the plates of a parallel plate capacitor
and imaged the drops with a microscope. Using the method described by the 'Balance Method'
section of the fundamental electron charge to be $\qel$ $\pm$ $\sel$ C. Other parameters
of the experiment had to be assumed, including the viscocity of air, $\eta$, which I took to be
$1.846 \times 10^{-5}$ $kg\  m^{-1}\  s^{-1}$. I assumed the atmospheric pressure was
$75.99$ $cmHg$. Calibration showed a $0.266\ mm$ wire was $15 \pm 0.5\ mm$ on the screen,
and a voltmeter showed the 'zero' on the apperatus was approximately $2\ V$. To find this value,
$17$ drops were characterized, and the charge on each was found using the 'Balance' method. Both
differences between charges were compared, and I made a histogram of the truncated values
with bins of $1 \times 10^{-20}\ C$ between the values of $1 \times 10^{-18}$ to $1 \times 10^{-20}$.
The reported value is the most populated bin. To improve the measurement, the terminal
velocity for each drop should have been measured more times. Accurate measurements of the atmospheric
pressure and temperature could small (almost negligible) effects on the reading, but would
tamp down uncertainty. More drops could be measured as well. Finally, the radii are calculated,
but I believe precise direct measurements of the droplet radii could improve accuracy of the measurement.

\end{abstract}
\end{document}
~
